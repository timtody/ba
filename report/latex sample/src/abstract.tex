\section*{Zusammenfassung}
Convolutional Neural Networks sind haben in den letzten Jahren in vielen Machine Learning Problemen bisher ungeahnte Benchmarks ermöglicht, die in einigen Domänen sogar Menschliche Performance übersteigen. Dabei sind rekurrente Verbindungen in solchen Netzen bisher selten beachtet worden, obwohl die Struktur des menschlichen Gehirns solche nahelegt. In dieser Arbeit versuche ich die Ergebnisse von Spoerer und Kriegeskorte {} zu reproduzieren, welche Hinweise fanden, dass das Inkorporieren solcher rekurrenten Verbindungen Vorteile bei der Bilderkennung in schwierigen Szenarien, wie der teilweisen Verdeckung zu erkennender Objekte bieten kann.

\begin{verbatim}

like whaat?

\end{verbatim}

\section*{Abstract}
