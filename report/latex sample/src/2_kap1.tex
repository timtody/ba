\section{Materialien und Methoden}\label{Materialien und Methoden}
\subsection{Generatives Modell für Stimuli}
Um die Effekte von Rekursion auf das klassifizieren von okkludierten Objekten zu untersuchen, wurde ein möglichst einfaches Grundproblem betrachtet. Die Klassifizierung von Ziffern ist ein gut untersuchtes Machine Learning Problem, in dem übermenschliche Performance erreicht werden kann. Allgemein gilt das sehr ähnliche MNIST-Datenset gemeinhin als das "Hello world!" des Machine Learning. Durch die Einfachheit der Aufgabe, können die Auswirkungen von Rekursion isoliert von anderen Herausforderungen betrachtet werden. 
\subsection{Modelle}
Um die Wirkungskraft von Rekursion zu untersuchen, wurden verschiedene CNNs mit variierenden Graden an Rekursion benutzt und systematisch miteinander verglichen. Die Nomenklatur richtet sich am Paper von Spoerer und Kriegeskorte aus. Als Grundlage dient eine Standard feed-forward Architektur (B) mit reinen 'bottom-up' Verbindungen. Da diese jedoch in der Anzahl der Parameter und der Anzahl der durchgeführten Konvolutionen gegenüber seinen Rekursiven Varianten unterlegen ist, wurden zum Vergleich zusätzlich Modelle entwickelt, die in den entsprechenden Domänen angepasst wurden. Einerseits wurde die Größe der Konvolutionskernel angepasst und somit die Größe der erlernbaren Features. Andererseits wurde die Anzahl der Konvolutionen erhöht, die Gewichte in den zusätzlichen Konvolutionen jedoch mit den anderen Konvolutionen geteilt. Somit kann die Anzahl der Faltungsoperationen vergleichbar gemacht werden, ohne die freien Parameter zu erhöhen. Die Archtitekturen werden im Folgenden BK respektive BKC genannt. 
\subsubsection{Implementierung}
