\section{Einleitung}\label{Einleitung}
Among a range of other tasks, image recognition has been a domain dominated by feed forward Convolutional Neural Network architectures. Strong evidence suggests, that the human brains ability to rapidly recognize objects under appearance variation is solved in the brain via a succession of largely feedforward computations. \cite[2012]{DICARLO2012415} Recent success of feedforward networks has only supported this hypothesis \cite{NIPS2012_4824}. However, evidence also suggests, that after the initial recognition process further processing takes place \cite{Sugase1999}. This delayed processing might indicate recurrent computations which I want to investigate in this work.
Using a range of convolutional architectures, I try to systematically measure the impact of recursion on image recognition tasks.
\\
Künstliches Intelligenz ist ein gedeihendes Gebiet mit zahlreichen Anwendungszwecken und einer überwältigenden Menge an aktiver Forschung. Ob autonomes Fahren, Pre-crime oder medizinische Diagnosen - Künstliche Intelligenz hat in vielen Gebieten für nie geahnte Erfolge gesorgt, bringt jedoch auch gewaltige soziale Implikationen mit sich. Maßgeblich verantwortlich für den Erfolg in vielen Gebieten sind vom Gehirn inspirierte Konzepte, deren Entstehung und Grundlagen ich im Folgenden erörtern werde.